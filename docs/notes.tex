\documentclass{article}

\newcommand{\vinf}{v_{\infty}}
\newcommand{\Rstar}{R_{\ast}}

\newcommand{\tr}{\tilde{r}}
\newcommand{\tv}{\tilde{v}}
\newcommand{\tti}{\tilde{t}}
\newcommand{\tell}{\tilde{\ell}}

\newcommand{\rin}{r_{\rm in}}
\newcommand{\rex}{r_{\rm ex}}
\newcommand{\trin}{\tilde{r}_{\rm in}}
\newcommand{\trex}{\tilde{r}_{\rm ex}}

\newcommand{\tchi}{\tilde{\chi}}
\newcommand{\tchistar}{\tilde{\chi}_{\ast}}

\newcommand{\ellstar}{\ell_{\ast}}
\newcommand{\tellstar}{\tell_{\ast}}

\newcommand{\taustar}{\tau_{\ast}}

\newcommand{\diff}[1]{{\rm d}#1}
\newcommand{\deriv}[2]{\frac{{\rm d}#1}{{\rm d}#2}}

\begin{document}

\section*{Smooth Wind Description}

Assume the wind is described by a $\beta$ velocity
law:
%
\begin{equation}
v(r) = \vinf \left( 1 - \frac{\Rstar}{r} \right)^{\beta},
\end{equation}
%
where $\vinf$ is the wind terminal velocity, $\Rstar$ is the stellar
radius, and $\beta$ is the velocity exponent. In terms of
dimensionless variables $\tv \equiv v/\vinf$ and $\tr \equiv r/\Rstar$,
this becomes
%
\begin{equation} \label{e:vel-law}
\tv(\tr) = \left( 1 - \frac{1}{\tr} \right)^{\beta}.
\end{equation}
%
The flow time to travel from some inner radius $\rin$ out to some radius
$r$ is
%
\begin{equation}
t(r) = \int_{\rin}^{r} \frac{1}{v(r')} \diff{r'},
\end{equation}
%
or, with $\tti \equiv t \vinf/\Rstar$ and $\trin \equiv \rin/\Rstar$,
%
\begin{equation} \label{e:trav-time}
\tti(\tr) = \int_{\trin}^{\tr} \frac{1}{\tv(\tr')} \,\diff{\tr'}.
\end{equation}
%
For general $\beta$, this integral cannot be evaluated analytically.

\section*{Clumed Wind Description}

Assume a wind in which the mass is confined to discrete, spherical
clumps, rather than being smoothly distributed. However, the velocity
of these clumps follows the same velocity law~(\ref{e:vel-law}). If
the solid angle subtended by the clumps (as viewed from the star)
remains constant, then the clump radius must evolve as
%
\begin{equation}
  \ell(r) = \ellstar \frac{r}{\Rstar}
\end{equation}
%
where $\ellstar$ is the radius at the stellar surface.

Let $\dot{n}$ be the clump passage rate --- the number of clumps per
unit time passing any given radius in the wind (\emph{which} radius
doesn't matter, as we're assuming clump numbers are conserved as they
flow outward). Then, the mass of each individual clump is
%
\begin{equation}
  m = \frac{\dot{M}}{\dot{n}}
\end{equation}
%
where $\dot{M}$ is the time-averaged wind mass loss rate. The density
of each clump (assumed uniform) follows as
%
\begin{equation}
  \rho(r) = \frac{m}{\frac{4\pi}{3} \ell^{3}} =
  \frac{3 \dot{M} \Rstar^{3}}{4\pi \dot{n} \ellstar^{3} r^{3}}.
\end{equation}
%
If $\kappa$ is the (assumed-constant) opacity of the wind material,
then $\chi \equiv \kappa\rho$ (the reciprocal of the in-clump mean
free path) becomes
%
\begin{equation}
  \chi(r) = \frac{3 \dot{M} \Rstar^{3}}{4\pi \dot{n} \ellstar^{3} r^{3}}.
\end{equation}
%
In dimensionless form, this is
%
\begin{equation}
  \tchi(\tr) \equiv \chi \Rstar = \frac{\tchistar}{\tr^{3}},
\end{equation}
%
where
\begin{equation}
  \tchistar \equiv \frac{3 \kappa \dot{M}}{4\pi \dot{n} \Rstar^{2} \tellstar^{3}}.
\end{equation}
%
This latter parameter is closely related to the wind optical depth parameter
%
\begin{equation}
  \taustar \equiv \frac{\kappa \dot{M}}{4 \pi \Rstar \vinf}
\end{equation}
%
representing the stellar optical depth assuming a smooth wind flowing
at $\vinf$. Specifically,
%
\begin{equation}
  \tchistar = \frac{3 \taustar}{\tilde{\dot{n}} \tellstar^{3}},
\end{equation}
%
where $\tilde{\dot{n}} \equiv \dot{n} \Rstar/\vinf$ is the dimensionless clump passage rate.

\section*{Numerical Implementation}

In this section, let's go over the details of setting up a clumpy
wind, as implemented in the \texttt{yarg\_wind} code. The code assumes
clumps are emitted at an inner radius $\rin$ via a random Poisson
process with rate $n$, and with a random angular position. Their
radius and optical thickness are set by $\ellstar$ and $\taustar$
parameters. The clumps advect outward with the wind, as described
above, until they reach an outer radius $\rex$, where they are
deleted. This means that the wind mass distribution extends only from
$\rin$ to $\rex$. In the code, all of these parameters are specified
in dimensionless (tilde) form.

To figure out where a given clump will be at a specific time, it's
necessary to invert the wind travel time
equation~(\ref{e:trav-time}). The code does this through a sequence of
steps:
%
\begin{itemize}
\item Set up a grid of velocities $\{\tv_{1}, \tv_{2}, \ldots,
  \tv_{N}\}$, with $\tv_{1} = \tv(\trin)$ and $\tv_{N} = \tv(\trex)$.
\item Evaluate a grid of corresponding radii $\tr_{i} \equiv \tr(\tv_{i})$ by
  inverting the velocity law~(\ref{e:vel-law}).
\item Construct a piecewise cubic function $f(\tr)$ such that
  \[
  f(\tr_{i}) = \frac{1}{\tv_{i}}
  \]
  and
  \[
  f'(\tr_{i}) = \left( -\frac{1}{\tv^{2}} \deriv{\tv}{\tr} \right)_{i}.
  \]
\item Setting $\tti_{1} = 0$, use the expression
  \[
  \tti_{i+1} = \tti_{i} + \int_{\tr_{i}}^{\tr_{i+1}} f(\tr') \,\diff{\tr'}
  \]
  to evaluate $\tti_{2}$, $\tti_{3}$, etc. The integrals are evaluated
  analytically, because $f(\tr)$ is a know cubic function over each
  interval $[\tr_{i},\tr_{i+1}]$.
\item Construct a piecewise cubic function $g(\tti)$ such that
  \[
  g(\tti_{i}) = \tr_{i}
  \]
  and
  \[
  g'(\tti_{i}) = \tv_{i}.
  \]
  This function serves as a good approximation to $\tr(\tti)$.
\end{itemize}
%

\end{document}
